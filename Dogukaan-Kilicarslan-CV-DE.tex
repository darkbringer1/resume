%-------------------------
% Resume in LateX
% Author : Sourabh Bajaj
% License : MIT
%------------------------

\documentclass[letterpaper,11pt]{article}

\usepackage{latexsym}
\usepackage[empty]{fullpage}
\usepackage{titlesec}
\usepackage{marvosym}
\usepackage[usenames,dvipsnames]{color}
\usepackage{verbatim}
\usepackage{enumitem}
\usepackage[hidelinks]{hyperref}
\usepackage{fancyhdr}
\usepackage[german]{babel}
\usepackage{tabularx}
\input{glyphtounicode}

\pagestyle{fancy}
\fancyhf{} % Clear all header and footer fields
\fancyfoot{}
\renewcommand{\headrulewidth}{0pt}
\renewcommand{\footrulewidth}{0pt}

% Adjust margins
\addtolength{\oddsidemargin}{-0.5in}
\addtolength{\evensidemargin}{-0.5in}
\addtolength{\textwidth}{1in}
\addtolength{\topmargin}{-.5in}
\addtolength{\textheight}{1.0in}

\urlstyle{same}

\raggedbottom
\raggedright
\setlength{\tabcolsep}{0in}

% Sections formatting
\titleformat{\section}{
  \vspace{-4pt}\scshape\raggedright\large
}{}{0em}{}[\color{black}\titlerule \vspace{-5pt}]

% Ensure that generate PDF is machine readable/ATS parsable
\pdfgentounicode=1

%-------------------------
% Custom commands
\newcommand{\resumeItem}[2]{
  \item\small{
    \textbf{#1}{: #2 \vspace{-2pt}}
  }
}

% Just in case someone needs a heading that does not need to be in a list
\newcommand{\resumeHeading}[4]{
    \begin{tabular*}{0.99\textwidth}[t]{l@{\extracolsep{\fill}}r}
      \textbf{#1} & #2 \\
      \textit{\small#3} & \textit{\small #4} \\
    \end{tabular*}\vspace{-5pt}
}

\newcommand{\resumeSubheading}[4]{
  \vspace{-1pt}\item
    \begin{tabular*}{0.97\textwidth}[t]{l@{\extracolsep{\fill}}r}
      \textbf{#1} & #2 \\
      \textit{\small#3} & \textit{\small #4} \\
    \end{tabular*}\vspace{-5pt}
}

\newcommand{\resumeSubSubheading}[2]{
    \begin{tabular*}{0.97\textwidth}{l@{\extracolsep{\fill}}r}
      \textit{\small#1} & \textit{\small #2} \\
    \end{tabular*}\vspace{-5pt}
}

\newcommand{\resumeSubItem}[2]{\resumeItem{#1}{#2}\vspace{-4pt}}

\renewcommand{\labelitemii}{$\circ$}

\newcommand{\resumeSubHeadingListStart}{\begin{itemize}[leftmargin=*]}
\newcommand{\resumeSubHeadingListEnd}{\end{itemize}}
\newcommand{\resumeItemListStart}{\begin{itemize}}
\newcommand{\resumeItemListEnd}{\end{itemize}\vspace{-5pt}}

%-------------------------------------------
%%%%%%  CV STARTS HERE  %%%%%%%%%%%%%%%%%%%%%%%%%%%%


\begin{document}

%----------HEADING-----------------
\begin{tabular*}{\textwidth}{l@{\extracolsep{\fill}}r}
  \textbf{\href{https://github.com/dogukaankilicarslan}{\Large Dogukaan Kilicarslan}} & E-Mail: \href{mailto:dogukaankilicarslan@gmail.com}{dogukaankilicarslan@gmail.com}\\
  \href{https://github.com/dogukaankilicarslan}{github.com/dogukaankilicarslan} & Mobil: \href{tel:+905326269925}{+90-532-626-9925} \\
  Berlin, Deutschland & \\
\end{tabular*}

%-----------EXPERIENCE-----------------
\section{Berufserfahrung}
  \resumeSubHeadingListStart

    \resumeSubheading
      {MigrosOne}{Remote}
      {iOS-Entwickler}{Dezember 2023 -- Aktuell}
      \resumeItemListStart
        \resumeItem{Migrationsprojekt}
          {Migrierte eine 1,4M-Zeilen-Codebasis von CocoaPods zu Swift Package Manager in 6 Wochen, reduzierte die Clean-Build-Zeit um 40\% und beseitigte Workspace-Merge-Konflikte.}
        \resumeItem{Architektur-Refactoring}
          {Refaktorierte den Checkout-Flow in 10 unabhängige MVVM-Module, ermöglichte parallele Entwicklung durch 3 Entwickler und reduzierte die PR-Zykluszeit um 30\%.}
        \resumeItem{Leistungsoptimierung}
          {Optimierte UIKit-Rendering-Hotspots, hielt tägliche Abstürze unter 0,1\% bei 880K täglich aktiven Nutzern und erhielt eine 4,8-App-Store-Bewertung.}
        \resumeItem{CI/CD-Pipeline}
          {Richtete Xcode Cloud + Bitrise-Pipelines ein, reduzierte Test-Build-Zeiten von 1-2 Tagen auf 2 Stunden.}
      \resumeItemListEnd

    \resumeSubheading
      {Hepsiburada}{Remote}
      {iOS-Entwickler}{September 2021 -- Dezember 2023}
      \resumeItemListStart
        \resumeItem{Nutzerwachstum}
          {Verdoppelte aktive Nutzer in 12 Monaten durch Bereitstellung von über 20 Features in einer Super-App für Verkäufer; verbesserte die Bewertung von 2,3 auf 4,7.}
        \resumeItem{Codequalität}
          {Entwickelte Legacy-Bildschirme zu VIPER um, verbesserte absturzfreie Sitzungen von 98,5\% auf 99,7\%.}
        \resumeItem{Entwicklungswerkzeuge}
          {Erstellte Bash-Skripte, die Modulvorlagen generieren und Coding-Standards durchsetzen.}
        \resumeItem{Leistungsprofilerstellung}
          {Leitete Leistungsprofilerstellung mit Instruments; behob zwei kritische Speicherlecks, reduzierte den App-RAM-Verbrauch um 50\%.}
      \resumeItemListEnd

    \resumeSubheading
      {Freiberuflich}{Remote}
      {Lead iOS Engineer}{Februar 2023 -- Aktuell}
      \resumeItemListStart
        \resumeItem{Architekturdesign}
          {Entwickelte ein SwiftUI-Produkt mit The Composable Architecture (TCA); leitete 3 Entwickler, führte Code-Owners \& automatisiertes Linting ein, erreichte absturzfreie Sitzungen von 99,8\%.}
        \resumeItem{Entwicklererfahrung}
          {Entwickelte benutzerdefinierte CLI-Tools zum Scaffolding von Features und zur Durchsetzung geschichteter neuer Module, reduzierte die Onboarding-Zeit für neue Mitwirkende um 80\%.}
      \resumeItemListEnd

  \resumeSubHeadingListEnd

%-----------EDUCATION-----------------
\section{Ausbildung}
  \resumeSubHeadingListStart
    \resumeSubheading
      {Boğaziçi Universität}{Istanbul, Türkei}
      {Master of Science in Software Engineering}{2025 -- Aktuell}
    \resumeSubheading
      {Atılım Universität}{Ankara, Türkei}
      {Bachelor of Science in Avionik-Wartungstechnik}{}
  \resumeSubHeadingListEnd


%-----------PROJECTS-----------------
\section{Führung \& Aktivitäten}
  \resumeSubHeadingListStart
    \resumeSubItem{Swift Buddies Community für iOS-Entwickler}
      {Gründete und entwickelte eine persönliche iOS-Community, veranstaltete entspannte Treffen, bei denen Entwickler aller Niveaus Wissen teilen, netzwerken und zusammenarbeiten. (August 2021 -- Aktuell)}
    \resumeSubItem{iOS Developers Community App}
      {Leitete eine von der Community betriebene App, die als Zentrum für Entwicklernetzwerke, Ressourcen und Veranstaltungen dient, iterierte Features basierend auf Nutzerfeedback. (Februar 2024 -- Aktuell)}
    \resumeSubItem{AWS/DevOps \& Go Development Bootcamp-Assistent}
      {Arbeitete während des Unterrichts mit dem Lehrer zusammen und übernahm die Anwesenheitskontrolle der Studenten. Nahm Sitzungen auf und lud sie hoch. Bewertete Hausaufgaben und betreute persönlich jeden Studenten. (April 2022 -- September 2022)}
    \resumeSubItem{Blah Blah English Speaking Club}
      {Mitbegründete und moderierte wöchentliche Online-Sitzungen für Nicht-Muttersprachler, wählte gemeinsam Themen aus und moderierte, um alle Teilnehmer einzubinden. (Juli 2020 -- Aktuell)}
  \resumeSubHeadingListEnd

%
%--------SKILLS------------
\section{Fähigkeiten}
  \resumeSubHeadingListStart
    \item{
      \textbf{Programmiersprachen}{: Swift, Python, Bash}
      \hfill
      \textbf{Frameworks}{: SwiftUI, UIKit, RxSwift, Combine, TCA}
    }
    \item{
      \textbf{Architektur}{: MVVM, VIPER, TCA, SOLID-Prinzipien, Design Patterns}
      \hfill
      \textbf{Werkzeuge}{: Xcode, Instruments, Git, CI/CD}
    }
    \item{
      \textbf{Sprachen}{: Englisch (C1), Türkisch (Muttersprache), Deutsch (Grundkenntnisse)}
    }
  \resumeSubHeadingListEnd


%-------------------------------------------
\end{document}

